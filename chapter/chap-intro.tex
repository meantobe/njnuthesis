
\chapter{引言}
\label{chap:introduction}

\section{什么是\TeX和\LaTeX}

我们先从一个很冷的、可能只有码农能找到笑点的笑话开始。

\begin{quotation}
Linus、Richard Stallman和Don Knuth一同参加一个会议。Linus说:“上帝说我创造了世界上最优秀的操作系统。” Richard Stallman自然不甘示弱,他说:“上帝说我创造了世界上最好用的编辑器。” Don Knuth一脸疑惑的说:“等等,我什么时候说过这些话?”
\end{quotation}

高德纳(Knuth的中文名)是历史上最年轻的图灵奖获得者,并且是靠写书拿奖的。这本书就是《计算机程序设计的艺术》(The Art of Computer Programming,简称TAOCP)\footnote{据说Bill Gates说过如下话语:“如果你认为你是一名真正优秀的程序员,就去读第一卷,确定可以解决其中所有的问题。”“如果你能读懂整套书的话,请给我发一份你的简历。”}。高德纳在写书时,发现当时的计算机排版质量很差,于是抛开了书,开始写一个排版程序\footnote{是不是很像处女座哈哈。}。这个排版程序,就是\TeX\footnote{书写时,三个字母都是大写,字母E应当低于其他两个字母。在不支持下标的系统中可写为“TeX”。},发音为“泰赫”。

\TeX 虽然强大,但是900多条命令使用起来太过麻烦。小说告诉我们,“但是”之后作者总是笔锋一转,大救星从天而降……好了,我们的救星就是Lamport,他为\TeX 编写了一组宏包(package)并命名为\LaTeX\footnote{同上。在不支持下标的系统中可写为“LaTeX”。}。从某种意义上来说,\TeX 类似于砖头,而\LaTeX~则像垒好的模型,将格式的细节隐藏在样式之后,成为目前最流行的科技写作工具之一\footnote{主要是数学、物理、计算机三个学科。}。

\section{什么样的论文适合用\LaTeX}

没有放之四海皆准的真理,不然金三胖……写论文也是这样。\LaTeX 是一门不算简单的语言,不是只要点点鼠标就能弄好的。那么,什么样的人适合使用\LaTeX 呢?

1. 有志于进入数学、物理和计算机这三个学科的学术界。

这三个领域中\LaTeX~比较占主流,许多期刊甚至只提供\LaTeX~模板。当然,国内某些期刊除外,呵呵。

2. 看见代码或者命令行不会反胃。

如果你见过、听说或者用过命令行、脚本、版本控制、程序设计语言等东西,并且在看到以上这些词汇后还没有反胃的话,作者推荐你使用\LaTeX 完成论文。

大多数人不喜欢编程,甚至敲一下dir就会晕倒过去。对于反胃的同学,作者只能摊双手表示同情,并推荐你读一下《禅与摩托车维修艺术》\footnote{http://book.douban.com/subject/6811366/}。